\section{Results}
This section quantifies findings after running the tool on various South Carolina elections.

\subsection{Missing Votes}

\subsection{Incomplete Audit Data}

\subsection{Polling Location Related Findings}

\subsection{Hardware Issues}
The machines used in South Carolina have experienced frequent potential hardware issues.  For example, the combination of machines in Berkeley County experienced votes cast on a machine when the machine was not calibrated, machines with possible low batteries, and at least one machine that closed early.  Our analysis found that there were seven counties where at least one machine was possibly not calibrated when votes were cast on that machine; these errors spanned 12 different polling locations.  We suggest an election official or technician inspect these machines for possible calibration issues.  We had similar findings when searching for terminals that recorded a "Warning - Terminal Closed Early" event.  There were machines with this warning in seven counties and 13 polling locations.  Terminals should not close before 7 P.M. in South Carolina on election day; for this reason, we recommend that these machines be evaluated for potential problems that would have caused early closure.  When our tool reports machines with possible low batteries, election officials should verify that the machine is working properly and does not need maintenance.  Florence County and Greenville county experienced a number of Internal Power Supply - related events; at least one machine in each precinct had 53 and 63 instances of this event, respectively.  This could be a possible indicator that the battery is running low; therefore, the election officials should take action to ensure all machines work properly in future elections. 

\subsection{Procedural Errors}
Our findings reveal the improvements needed for poll worker training and for various procedures.  Although we found that pollworkers were following procedures concerning the printing of zero tapes, there were a number of counties with procedural errors.  When opening and closing a machine, the same master PEB should be used, but in 11 counties there were cases of opening and closing machines with different PEBs.  Our results showed a correlation between this error and certain precincts, where pollworkers made those mistakes repeatedly.  Colleton County had five instances of this procedural error, but four of those instances took place at one polling location; Walterboro No 4 had machines 5129946, 5133679, 5138439, and 5138563 opened with PEB 155914, but closed with PEB 155925.  This should raise a red flag to the election officials that they may need to emphasize this procedure in poll worker training.  When poll workers activate ballots for voters, they should do so with a non-master PEB; we saw two counties that had an unusually high number of violations of this procedure.  Horry County and Richland County had 22 and 32 instances of this violation, respectively.  When election officials see this result, they may wish to revise poll worker training.  Our tool also analyzes the reasons why votes were canceled, which could give insight to procedural errors.  There is likely to be a certain number of vote cancellations due to a number of reasons, but our tool will only report the machines that recorded an abnormally large amount of vote cancellations for a specific reason.  Colleton County had a machine that recorded 12 instances of vote cancellations due to a terminal problem; in this case, we would recommend the officials to inspect the machine for potential hardware problems.  A machine in Lexington County experienced an unusually large number of vote cancellations due to a "wrong ballot"; this could be result of many problems.  The machine may have a calibration issue, or there may be a procedural error in that the poll workers are repeatedly selecting the wrong ballot.

\subsection{Date/Time Errors}
