\subsection{Systematic Date and Time Errors}\label{an:date}
The iVotronic DREs append each audit event in chronological order.  Each event is marked with a time-stamp based on the DRE's internal clock.  We discovered and report on a variety of errors by classifying the type of errors experienced. Correct time-stamps are critical in post election audits and often incorrect stamps can not be automatically corrected post election.  Previous works identify and remark on some of these date errors~\cite{Buell2011,Sandler2007}.  We further attempt to classify and automate the identification of these issues.

Erroneous time-stamps can invalidate the audit-logs and often preclude data from being used in automated reports.  Determining machines to have either valid or invalid time-stamps has a lot of gray area and different errors will affect different analyses.  Some machines experienced time-stamps that would blank to \textquoteleft 00/00/00 00:00:00' for only a couple of events.  This naturally wouldn't affect data looking at opening and closing times, but would create outliers or gaps in data measuring time between votes cast.  

We found it simplest to classify date errors into two categories.  Those errors resulting from machines not having their clocks set appropriately and those resulting from apparent bugs in the iVotronic time-stamp mechanism itself.  Our website includes an automated a report that attempts to identify and group as many of these errors as possible.

First, machines shipped or set with incorrect dates were identified.  These errors all suggest the need for a more thorough pre-election check of each machine's clock.   Any machines experiencing manual clock adjustments on election day were included in the report as well as those machines which opened for voting on a date that was wildly incorrect (i.e. dates well before pre-election or dates after election day). This mostly included machines that did not account for Daylight Saving Time or  those machines that didn't set their initial date until after opening for voting.  All the above machines were checked as closing on a valid election day.  Pre-voting dates were not considered as they appeared to be inconsistent among the different counties. Machines which open and close on an improbable dates are separately identified as machines that had bad dates that went uncorrected..  

Machines experiencing additional date errors were classified in a separate report.  Many machines were found to have anomalous date changes that weren't paired with the normal date set event. [Figure 1 will show the 4/12 jump in Berkeley county] Often before the clock on a machine is first set the dates will show up as being many years into the future or as a zero date.  This isn't a problem as most  of these machines are set correctly before being opened for voting.  However, there are many places in the logs were the date will seemingly randomly jump to a date far into the future or the past and remain there until manually corrected.  [Figure 2 12\/21 in Berk County] shows a case where the date jumps ahead to 12/21/2010 for two events before changing back.  These machines were automatically identified by looking for any major date jumps that occur on election day or zero stamps being recorded after machines are open. Machines experiencing many date jumps may require troubleshooting from ES\&S.  It may be a more systematic issue that the date on an iVotronic can apparently change for no reason.

We strongly advise that procedures for setting the clocks on machines are reviewed.  The unknown date jumps seen in the logs are concerning, but generally are not creating as many audit issues compared to machines whose date were never set correctly.

Accurate clocks directly impact the usefulness and correctness of the audit logs. Ensuring that every single machine is set correctly is not necessarily a simple task.  We would recommend that each machine is configured accurately before being sent to the precincts.  Additionally, all machines should be double checked for a correct time before opening for voting.  Daylight Saving Time settings are also potential concern.  Many machines (statistic for anderson) were found to be adjusted forward by an hour during election day.  

The Authors of \textquotedblleft Casting Votes in the Auditorium\textquotedblright~\cite{Sandler2007} propose a distributed network between DREs.  This \textquoteleft Auditorium\textquoteright provides a far more robust system to ensure accurate and verifiable audit logs.  They propose a system where all the election machines are networked together and append to a common audit log verified by each machine.  This allows for more error redundancy and removes the logistical issue involved in making sure every single machine has their date correctly set.
