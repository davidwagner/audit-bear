\subsection{Polling Location Related Analyses}
\subsubsection{Long Lines}
Election officials assign voting machines and polling location supplies based on the number of voters registered in each precinct.  As a result, some polling locations may end up overstocked with equipment, precinct supplies or poll workers while others may lack resources or personnel on election day. Monitoring all the polling places in a large county can be a daunting task. Often, election officials don't have any process in place to monitor polling location usage. South Carolina counties have experienced voting machine bottlenecks during the 2008 and 2010 elections~\cite{Kreitman2010, Slade2008, U2010}.  Those counties can benefit from a tool that can analyze DRE audit data to identify peak times at the precincts.  Our study can infer a steady flow of voters from two iVotronic log files (EL152 and EL155) and produce a report detailing possibly busy timeframes due to the number of voting machines. Such information will assist election officials with the planning of future elections by augmenting voting machines or resources where they may be lacking.

Our tool focuses on lines of voters by detecting heavily used voting terminals. When there are consecutive ballots cast with no time delay in between, we are able to infer that there is a line of voters at the voting machines. 

Once our tool groups the iVotronic units by polling location, based on the information contained in the ballot images report (EL155), it determines that all the machines in the voting location were in use. This analysis also uses a function that finds polling locations which closed late in addition to a time/date verification function that invalidates and excludes voting machines with anomalous date/time settings as described in section~\ref{an:date}. The date and time of iVotronic terminals is set manually and subject to human error; therefore creating inaccuracy in the timing of the events identified by the audit log.  The time verification function is significant in determining which machines were heavily used at the precinct in a specific time period; therefore inferring the possibility of long lines at the precinct.

To infer long lines, we focused on the polling locations that stayed opened after 7 P.M. as we could conclude they were busy processing the voters standing in line at that time. Our analysis calculates the time between consecutive votes before 7 P.M.; we keep track of the time that these votes occurred and the time difference between votes.  For all of the consecutive votes after 7 P.M., we only store the time difference between votes.  This data is found per machine, which allows us to match it to its respective polling location.  Then, we organize the time differences into one-hour time windows starting at 7 A.M. until 7 P.M.; all of the after-7 P.M. data was grouped together.  Then, focusing on the polling locations that close very late, we use the two sample Kolmogorov-Smirnov test to determine whether the votes cast in a particular time window come from the same distribution as the after-7 P.M. votes.  The result of the statistical test returns two values; one of them is the p-value. A p-value less than 10\%, indicates the two samples are unlikely come from the same distribution, and therefore there probably weren't long lines.  Otherwise, a p-value higher than 10\% is consistent with the two samples coming from the same distribution, however, we can not make any concrete conclusions based on a high p-value, we can only note that it is possible there were long lines in that polling location.

%which is the probability that the two sample data have the same distribution.  This p-value helped us establish whether a polling location had long lines at a certain time window.  If the p-value is more than 10\% then we can infer that the polling location possibly had long lines in that time window.
%Fix this part (add a table precinct name, p-value is less than 10%, p-value > 10%), then explain the table.
In Berkeley County, we found that 17 precincts were closed after 7:30 P.M. and decided to run the analysis in these locations to determine when there were long lines.  We will emphasize the top five precincts that close very late.  Our analysis reveals that the first precinct Huger \#26, which closed at 8:43:44 P.M., has higher p-values at 7:00 A.M., 8:00 A.M., 10:00 A.M., 11:00 A.M., 1:00 P.M., 2:00 P.M., 3:00 P.M. and 5:00 P.M..  The minimum p-value is 15.4\% at 2:00 P.M. and the maximum is 66.0\% at 11:00 A.M.  Precinct Cordesville \#10 could possibly have had long lines throughout almost the entire day, except for 7:00 A.M. and 12:00 P.M. with a minimum p-value of 13.6\% at 10:00 A.M. and the rest of the time had p-values higher than 20\%.  Precincts Hilton Cross Rd \#24, and Hanahan 1 \#20 also had long lines almost all day.  Precinct Hanahan 3 \#22 only experienced long lines at 5:00 P.M. with a p-value of 38.5\%.  We can conclude that these precincts have experienced long lines during the whole day and it may be the reason for which those precincts closed very late.

%\begin{table*}
%    \begin{center}
%    \begin{tabular}{| l | c | c |}
%    \hline                   
%    \# Precint &Probably weren't long lines (p-value \textless 10\%)&Possibly long lines (p-value \textgreater 10\%)\\
%    \hline
%    26 Huger&9:00 A.M.&7:00 A.M. \\
%    \hline
%    7 & 8 & 9 \\
%    \hline  
%    \end{tabular}
%    \end{center}
%    \caption{An example of table}
%    \label{my_table}
%\end{table*}

\begin{tabular}{|l||l|l||l|l|}
\hline
 &\multicolumn{2}{l|}{Singular}&\multicolumn{2}{l|}{Plural}\\
\cline{2-5}
 &English&\textbf{Gaeilge}&English&\textbf{Gaeilge}\\
\hline\hline
1st Person&at me&\textbf{agam}&at us&\textbf{againn}\\
2nd Person&at you&\textbf{agat}&at you&\textbf{agaibh}\\
3rd Person&at him&\textbf{aige}&at them&\textbf{acu}\\
 &at her&\textbf{aici}& & \\
\hline
\end{tabular}

We strongly recommend election officials to conduct this analysis after each election to plan for future elections of the same type. Assigning additional personnel, whether poll workers or rovers, and machines to the busy polling locations may reduce long lines of voters.  Our analysis can detect when there is a steady flow of voters, but it does not determine if the long line of voters is caused by a slow registration process or too few voting machines.

\subsubsection{Polling Locations That Closed Late}
[To complete by Wednesday].
