\section{Analyses}
\subsection{Votes Possibly Not Uploaded}
One of the most important aspects of any election audit is ensuring that all votes are counted.  There are two main pieces of the election system that need to be analyzed to determine if votes were left out of the count.  The first is the PEB; keeping track of what happens to the PEB can tell us not only if the votes were uploaded, but from which machines those votes came from.  This is done by comparing the audit files produced by the iVotronic.  The system log file can be crosschecked with every PEB used to close a machine in order to determine whether the votes on that PEB were uploaded.  The tool we implemented will display the user with the precinct name and number, PEB number, machine serial number, and the number of votes that were held on that PEB.  The second piece of the election system that must be taken into account is the voting machine itself.  If we could not conclude lost votes from the PEB, we may be able to infer them from the machines' event logs.  If a machine is not closed, then a PEB has not collected this terminal's data.  By checking the event log, we can determine which machines had not been closed; our tool will display the precinct name and number, and the machine serial number that was not closed. 

Ensuring all votes are counted is one of the most important audits that can be done on election data.  The goal of an election is to decide on a candidate with the input of everyone that has voted.  This analysis is important because it helps detect the cases where some votes were not counted, and gives officials enough information to collect those votes.  This type of audit is especially useful when the margin between candidates is extremely small.  In the case that uncollected votes go unknown, the wrong candidate can win an election.  

While detecting these analyses will aid officials in finding votes that may have been lost, they cannot guarantee to find all uncounted votes.  Depending on the circumstances, some votes will remain uncounted because the event logs are not suitable to other auditing techniques.  In order to account for more missing votes, a list of machines used in each precinct would be extremely helpful.  With the files used by most of our analyses, we assume audit data is complete; by conducting a different analysis, we know that this is not true.  When the audit logs are incomplete we cannot account for every machine, thus we cannot ensure all votes are counted.  If there was a master list of machines used in each precinct, then there would no longer be a problem of keeping track of all machines.  
 
The combination of audit logs produced by the iVotronic reveals surprising information about votes that had mistakenly been left out of the certified count. [Examples from South Carolina here] [Our suggestions to officials here].

\subsection{Incomplete Audit Data}
In order to conduct an accurate audit of an election, the audit data must include everything that was recorded on all voting machines during the election.  One of our analyses check both the event log and ballot images to see if the appropriate machines are present.  Not only is it important that there are ballot images for every machine with votes cast on it, but we need to verify that there is an equal number of vote cast events and ballot images per machine.  When a machine occurs in the event log with recorded votes cast on it during election day, but does not appear in the ballot images, then there must be data missing from the ballot images file.  The reverse situation reveals the opposite error- the event log is not complete.  

The importance of complete audit data lies in the accuracy of auditing elections.  In South Carolina, the only components of the election paper trail are the audit logs; this magnifies the importance of having complete information.  While we can identify missing information from the event log and the ballot images file, we assume the data is complete when conducting our other analyses.  If a county supplies an incomplete log, our tool's results will be less accurate than they would be otherwise.  This is problematic because there is a higher possibility for leaving anomalies undetected; missing votes may not be found and officials may not be able to identify why errors occurred during the election.  

While the audit data is clearly not always complete, there are other cases where it may be nearly impossible to tell if the data is incomplete or not.  For example, if a machine was opened on election day, experienced severe problems, had no votes cast on it, and was not included in the event log, it would be undetectable unless the voting system was altered; the creation of a list of machines used at each precinct would benefit this analysis in addition to detecting possibly missing votes.  In the case that the files are incomplete, we assume it to be a result of uploading vote data into the database at different times.  Because there are two different databases (one for the event log and one for the ballot images), the system allows for new data to be uploaded between the creation of the two files.  If there were one database used for both logs, this would reduce the problem.  On the other hand, if there was just one database, it would be nearly impossible to detect incomplete data.    

[Include table as example of this analysis] [Finding from our analyses here] [Suggestions for officials here].

\subsection{Long Lines/ Locations opened/closed late}

\subsection{Hardware Issues}
Election officials may be interested in identifying machines that have hardware problems, such as screen calibration issues, machines with a low battery, terminals that closed early, and machines that recorded unknown, but possibly severe events.  The first of these analyses detects machines with recurring calibration errors and machines that had recorded votes while possibly not calibrated.  By finding the events that correspond to a screen that is not calibrated and to the recalibration of that screen, we can find if votes were cast in between those times.  The second analysis regarding hardware issues looks for machines with an unusually large number of events titled "Terminal shutdown - IPS Exit."  We infer that these machines have a low battery because they experience a more-than-normal number of events related to the Internal Power Supply.  Additionally, our tool searches for machines that recorded a warning event about the terminal closing early.  In order for this event to occur, an election worker (what is the name of this person?) must enter a password to access the service menu and make a particular selection to close the machine.  If a machine is closed in this manner during election day, there must be something wrong with it that is preventing votes from being cast correctly.  Lastly, there is a set of events that have questionable meanings, but could potentially represent hardware issues.   

Analyses such as these can help officials identify machines that may require maintenance or to be replaced.  In the case of a machine having ballots cast on it when it is not calibrated, it may not have captured the voter's intent.  Depending on the magnitude of the situation, this could cause a different outcome in the election.  If our analyses detect other hardware problems with a machine, it may not be recording votes accurately; these votes may not even appear in the event log or ballot images.  If the event log and the ballot images do not record ballots being cast, then it is nearly impossible for officials to realize votes are not being counted.    

Due to the available resources and the nature of these analyses, assumptions were made regarding the meaning of events and the severity of the situation.  Currently, there is no user manual or detailed description of the events that appear in the event log; because of this, we are not able to guarantee that the event "Terminal shutdown - IPS Exit" means the machine has a low battery.  This assumption is also applicable to our calibration analysis and our detection of unknown warning events.  If a description of each event was available, we could be more definitive in our results and possibly implement analyses that report other useful hardware failures.  

[Statistics from reports here] [Suggestions/ corrective actions to take].  

\subsection{Procedural Errors}
Our tool can detect procedural errors and poll worker mistakes; a few of these are: precincts that do not print zero tapes on the morning of election day, using a master PEB to activate ballots, opening and closing machines with different PEB.  According to the South Carolina poll worker training video (citation), poll workers are required to print at least one zero tape per polling location on the morning of the election.  Using the event log, our tool checks each polling location for this event and reports the locations that did not record this event.  Another way our tool finds procedural errors is by crosschecking the master PEBs with the PEBs used to activate ballots.  Poll workers should be using non-master PEBs to activate ballots so that the PEBs do not get switched.  Along the same lines, we report incidents of opening and closing a machine with different PEBs.  A machine should be opened and closed with the same master PEB; if not, it may be more likely that this PEB does not get uploaded.   When poll workers cancel ballots, they must select a reason why; this is another way to detect errors.  There are seven options for canceling a ballot: wrong ballot, voter left after the ballot was issued, voter left before the ballot was issued, voter request, printer problem, terminal problem, or an unspecified reason.  If there are any instances of canceling a ballot due to a printer problem, that could be an indicator of a procedural error because ballots are not printed.  In other cases, if there is a large number of a specific reason, such as having the wrong ballot, this could indicate the poll workers are repeatedly issuing the wrong ballot.  

It may be beneficial to election officials if they could detect which locations'' poll workers are following the required procedures in order to mitigate larger problems.  If election officials are aware of the procedures that are not being followed, they will hopefully be able to mitigate them.  This will allow for more efficient audits as well as a better voting experience for voters.  Procedural errors can cause many problems including lost votes, incorrect vote counts, disgruntled voters, and long lines.

While our analyses detect an important set of errors, there are certainly many more procedures that can be analyzed.  In addition to printing zero tapes in the morning, poll workers are required to print results tapes at the end of the election; unfortunately, this is not detectable due to the way the event log is produced.  We have inferred from the event logs that the poll workers are extracting the compact flashes before printing the results tapes, therefore the event log shows no record of the event.  

[Statistics from reports here] [Suggestions/ corrective actions to take].  

