\section{Analyses}
This section presents a description of our analyses and important findings.
\subsection{Votes Possibly Not Counted}
\subsubsection{PEBs Not Uploaded}
Currently, the election officials performing the vote tabulation on election night do not have a way to effectively compare the list of PEBs that have been successfully uploaded to the tabulation software with the pool of PEBs containing vote data throughout the county. This can lead to inaccuracies in election results if any PEBs containing vote data are accidentally left out of the tally process. Our tool provides an analysis that generates a list of PEBs used to collect votes on election day. It warns election officials of any PEB, master or non-master, that was used to close a terminal, whose data wast not uploaded to the election reporting system.   

The precinct procedures for poll workers dictate that a single PEB should be used to open and close all machines at a polling location. Failure to strictly follow this protocol led to problems in Miami-Dade County during the 2002 Primary election~\cite{Mazella2002}. In that case, poll workers used two or more PEBs to open and close terminals at their precinct.  However, election workers at county headquarters only uploaded one of these PEBs, expecting all machines to be closed with the same PEB. This caused a significant delay in the reporting of election results. A recent study found similar problems in several South Carolina counties~\cite{Buell2011}.

This analysis is specific to the iVotronic system. It requires the event log, the ballot image file, and the system log. The event log records the serial number of the PEB used to close the terminal.  It also records, in chronological oder, vote events processed in the voting machine. The system log file tracks a running log of the serial number of PEBs uploaded to the ERM tabulation system. The ballot image file provides a list of terminals used in each precinct.

Our method keeps track of the serial number of the PEB used to close each voting machine. It also records the polling location each voting machine was assigned to as well as the total votes cast in each machine. Then,  it verifies that each PEB that was used to close voting machines is uploaded to the ERM vote tally. It reports any PEB containing vote data that has not been added to the cumulative count.  For each such PEB, our analysis reports the serial numbers of the terminals collected by the PEB, the number of votes processed in those terminals and the precinct's name and number. With this information, election officials can gather the missing PEBs and collect votes from terminals not included in the cumulative totals.

\subsubsection{Machines Not Closed}
There are two main pieces of the election system that need to be analyzed to determine if votes were left out of the count.  We have discussed the first essential piece: the PEB.  The second piece of the election system that must be taken into account is the voting machine itself.   We have devised a way to determine which machines have not been closed for voting.  If a machine is not closed, then a PEB has not collected this terminal's data; our algorithm will help detect this situation where some votes were not counted.  

This analysis uses the event log and the ballot image file to provide feedback; the event log should have the ability to record events marking the opening and closing of each voting machine.  The ballot image file allows us to identify which machines were at each polling location.  We created a method that checks if a machine was closed, given it was also opened for voting.  Our analysis searches through the event log for machines that were opened; these machines are stored in a data structure.  Next, we check that every machine in the data structure also recorded an event representing its closure.  If there are any machines that have been opened and not closed, they are displayed to the election official.   

\subsection{Polling Location Related Analyses}
\subsubsection{Polling Locations that Closed Late}
In the United Stated, poll hours are regulated by the state
officials. Poll closing hours vary from 6pm to 9pm depending on the
state~\cite{Info2007}. Some state statutes allow the voters waiting in
line at poll closing time to cast their ballots at the precinct;
therefore, polling locations may stay open late in order to
accommodate those voters waiting in line at the official poll closing
time. If election officials knew which polling places were likely to
experience long lines they could deploy more equipment or personnel to
those locations. This analysis can assist them by providing
information about long lines that occurred in the current election. Election
officials can use this information to make predictions about where
long lines might occur in future elections of the same type. 

This analysis gathers the time each machine was closed from the event log and the precinct the machine was assigned to from the ballot image file; to perform the analysis, this data is required.  Our algorithm saves this information in a data structure and generates a countywide chart detailing the number of polling locations that stayed open after poll closing time and for how long. To handle inaccurate machine date/time settings, our tool uses a time verification algorithm described in Section 3.6 to exclude from our database any voting machines whose time stamp is probably incorrect. 

\subsubsection{Long Lines}
Election officials assign voting machines and supplies to each polling
location based on the number of voters registered in the precinct.
However, voter turnout can vary and as a result, some polling
locations may end up overstocked with equipment, supplies or poll
workers while others may lack resources on election day. As a result,
voters may have to wait in line before casting their vote. This is
common during mid-term and general elections~\cite{Kreitman2010,
  Slade2008, U2010}, as well as any elections with high-voter turnout.
In those circumstances, election officials might like information
about which locations experienced long lines and at what time of the
day. However, monitoring all the polling places in a large county can
be a daunting task. Therefore, we provide an analysis that can analyze
DRE audit data to identify peak times at the precincts. Such
information can assist election officials with the planning of
resource allocation for future elections.

This analysis reports busy locations by detecting heavily used voting terminals. It requires an event log that captures every vote cast event along with an accurate time stamp and precinct location information. In our analysis we find this information in the event log and the ballot image file. When there are consecutive ballots cast in all machines in a polling location with no time delay in between, we are able to infer that there is a steady flow of voters and possibly a line of voters waiting to use the voting machines. 

However, the analysis is not quite so straightforward. The event logs
we used captured the \textquotedblleft cast vote'' event, but that
only tells us when the voter finished making their selection. In order
to know if there is any idle time between voters, we would also need
an event that notes when a new ballot is loaded into a machine. The
idle time would be the time between one \textquotedblleft cast vote''
event and the next \textquotedblleft ballot loaded''
event. Unfortunately our logs have no such \textquotedblleft ballot
loaded'' event. Instead, we had to infer the idle time. We did this by
focusing on the polling locations which stayed opened after poll
closing time as we could conclude they were busy processing the voters
standing in line at that time and the time between cast vote events
includes no idle time. For those locations, we measure the time it
took voters to cast ballots during the extended poll hours. We also
calculate and keep track of the time per ballot cast during regular
poll hours. Using the Kolmogorov- Smirnov statistical test we can
determine whether the distribution of time between vote cast events
during regular poll hours, in one-hour time windows, matches the
distribution of time between vote cast events during the extended poll
hours. If the distribution of the two samples is consistent, we can
infer the possibility of long lines during the particular one-hour
time period. \footnote{The KS test starts with the null hypothesis
  that the two samples come from the same distribution, which is the
  situation we are interested in identifying. Therefore, the most we
  can say in the case that the null hypothesis cannot be rejected is, we
  cannot reject the possibility that there were long lines between a
  particular time window.} 


\subsection{Hardware Issues}
Election officials may be interested in identifying machines that have hardware problems, such as screen calibration issues, machines with a low battery, terminals that closed early, and machines that recorded unknown, but possibly severe events.  The first of these analyses detects machines with recurring calibration errors and machines that had recorded votes while possibly not calibrated.  By finding the events that correspond to a screen that is not calibrated and to the recalibration of that screen, we can find if votes were cast in between those times.  The second analysis regarding hardware issues looks for machines with an unusually large number of events titled \textquotedblleft Terminal shutdown - IPS Exit\textquotedblright .  We infer that these machines have a low battery because they experience a more-than-normal number of events related to the Internal Power Supply.  Additionally, our tool searches for machines that recorded a warning event about the terminal closing early.  In order for this event to occur, a trained technician must enter a password to access the service menu and make a particular selection to close the machine.  If a machine is closed in this manner during election day, there must be something wrong with it that is preventing votes from being cast correctly.  Lastly, there is a set of events that have questionable meanings, but could potentially represent hardware issues.  

Analyses such as these can help officials identify machines that may require maintenance or to be replaced.  In the case of a machine having ballots cast on it when it is not calibrated, it may not have captured the voter's intent.  Depending on the magnitude of the situation, this could cause a different outcome in the election.  If our analyses detect other hardware problems with a machine, it may not be recording votes accurately; these votes may not even appear in the event log or ballot images.  If the event log and the ballot images do not record ballots being cast, then it is nearly impossible for officials to realize votes are not being counted.  

Due to the available resources and the nature of these analyses, assumptions were made regarding the meaning of events and the severity of the situation.  Currently, there is no user manual or detailed description of the events that appear in the event log; because of this, we are not able to guarantee that the event \textquotedblleft Terminal shutdown - IPS Exit\textquotedblright means the machine has a low battery.  This assumption is also applicable to our calibration analysis and our detection of unknown warning events.  If a description of each event was available, we could be more definitive in our results and possibly implement analyses that report other useful hardware failures.    

[Statistics from reports here] [Suggestions/corrective actions to take].

\subsection{Procedural Errors}
Our tool can detect procedural errors and poll worker mistakes; a few of these are: precincts that do not print zero tapes on the morning of election day, using a master PEB to activate ballots, opening and closing machines with different PEB.  According to the South Carolina poll worker training video (citation), poll workers are required to print at least one zero tape per polling location on the morning of the election.  Using the event log, our tool checks each polling location for this event and reports the locations that did not record this event.  Another way our tool finds procedural errors is by crosschecking the master PEBs with the PEBs used to activate ballots.  Poll workers should be using non-master PEBs to activate ballots so that the PEBs do not get switched.  Along the same lines, we report incidents of opening and closing a machine with different PEBs.  A machine should be opened and closed with the same master PEB; if not, it may be more likely that this PEB does not get uploaded.   When poll workers cancel ballots, they must select a reason why; this is another way to detect errors.  There are seven options for canceling a ballot: wrong ballot, voter left after the ballot was issued, voter left before the ballot was issued, voter request, printer problem, terminal problem, or an unspecified reason.  If there are any instances of canceling a ballot due to a printer problem, it could be an indicator of a procedural error because ballots are not printed.  In other cases, if there is a large number of a specific reason, such as having the wrong ballot, this could indicate the poll workers are repeatedly issuing the wrong ballot.  

It may be beneficial to election officials if they could detect which locations poll workers are following the required procedures in order to mitigate larger problems.  If election officials are aware of the procedures that are not being followed, they will hopefully be able to mitigate them.  This will allow for more efficient audits as well as a better voting experience for voters.  Procedural errors can cause many problems including lost votes, incorrect vote counts, disgruntled voters, and long lines.

While our analyses detect an important set of errors, there are certainly many more procedures that can be analyzed.  In addition to printing zero tapes in the morning, poll workers are required to print results tapes at the end of the election; unfortunately, this is not detectable due to the way the event log is produced.  We have inferred from the event logs that the poll workers are extracting the compact flashes before printing the results tapes, therefore the event log shows no record of the event.  

[Statistics from reports here] [Suggestions/corrective actions to take].

\subsection{Systematic Date and Time Errors}\label{an:date}
The iVotronic DREs append each audit event in chronological order.  Each event is marked with a time-stamp based on the DRE's internal clock.  We discovered and report on a variety of errors by classifying the type of errors experienced. Correct time-stamps are critical in post election audits and often incorrect stamps can not be automatically corrected post election.  Previous works identify and remark on some of these date errors~\cite{Buell2011,Sandler2007}.  We further attempt to classify and automate the identification of these issues.

Erroneous time-stamps can invalidate the audit-logs and often preclude data from being used in automated reports.  Determining machines to have either valid or invalid time-stamps has a lot of gray area and different errors will affect different analyses.  Some machines experienced time-stamps that would blank to \textquoteleft 00/00/00 00:00:00' for only a couple of events.  This naturally wouldn't affect data looking at opening and closing times, but would create outliers or gaps in data measuring time between votes cast.  

We found it simplest to classify date errors into two categories.  Those errors resulting from machines not having their clocks set appropriately and those resulting from apparent bugs in the iVotronic time-stamp mechanism itself.  Our website includes an automated a report that attempts to identify and group as many of these errors as possible.

First, machines shipped or set with incorrect dates were identified.  These errors all suggest the need for a more thorough pre-election check of each machine's clock.   Any machines experiencing manual clock adjustments on election day were included in the report as well as those machines which opened for voting on a date that was wildly incorrect (i.e. dates well before pre-election or dates after election day). This mostly included machines that did not account for Daylight Saving Time or  those machines that didn't set their initial date until after opening for voting.  All the above machines were checked as closing on a valid election day.  Pre-voting dates were not considered as they appeared to be inconsistent among the different counties. Machines which open and close on an improbable dates are separately identified as machines that had bad dates that went uncorrected..  

Machines experiencing additional date errors were classified in a separate report.  Many machines were found to have anomalous date changes that weren't paired with the normal date set event. [Figure 1 will show the 4/12 jump in Berkeley county] Often before the clock on a machine is first set the dates will show up as being many years into the future or as a zero date.  This isn't a problem as most  of these machines are set correctly before being opened for voting.  However, there are many places in the logs were the date will seemingly randomly jump to a date far into the future or the past and remain there until manually corrected.  [Figure 2 12\/21 in Berk County] shows a case where the date jumps ahead to 12/21/2010 for two events before changing back.  These machines were automatically identified by looking for any major date jumps that occur on election day or zero stamps being recorded after machines are open. Machines experiencing many date jumps may require troubleshooting from ES\&S.  It may be a more systematic issue that the date on an iVotronic can apparently change for no reason.

We strongly advise that procedures for setting the clocks on machines are reviewed.  The unknown date jumps seen in the logs are concerning, but generally are not creating as many audit issues compared to machines whose date were never set correctly.

Accurate clocks directly impact the usefulness and correctness of the audit logs. Ensuring that every single machine is set correctly is not necessarily a simple task.  We would recommend that each machine is configured accurately before being sent to the precincts.  Additionally, all machines should be double checked for a correct time before opening for voting.  Daylight Saving Time settings are also potential concern.  Many machines (statistic for anderson) were found to be adjusted forward by an hour during election day.  

The Authors of \textquotedblleft Casting Votes in the Auditorium\textquotedblright~\cite{Sandler2007} propose a distributed network between DREs.  This \textquoteleft Auditorium\textquoteright provides a far more robust system to ensure accurate and verifiable audit logs.  They propose a system where all the election machines are networked together and append to a common audit log verified by each machine.  This allows for more error redundancy and removes the logistical issue involved in making sure every single machine has their date correctly set.
