\subsection{Incomplete Audit Data}
In order to conduct an accurate audit of an election, the audit data must include everything that was recorded on all voting machines during the election.  One of our analyses checks both the event log and ballot images to see if the appropriate machines are present.  Not only is it important that there are ballot images for every machine with votes cast on it, but we need to verify that there is an equal number of vote cast events and ballot images per machine.  When a machine occurs in the event log with recorded votes cast on it during election day, but does not appear in the ballot images file, then there must be data missing from the ballot images file.  The reverse situation reveals the opposite error- the event log is not complete.  

The importance of complete audit data lies in the accuracy of auditing elections.  In South Carolina, one of the few components of the election paper trail are the audit logs.  While we can sometimes identify missing information from the event log and the ballot images file, we assume the data is complete when conducting our other analyses.  If a county supplies an incomplete log, our tool's results will be less accurate than they would be otherwise.  This means anomalies may go undetected; missing votes may not be found and officials may not be able to identify why errors occurred during the election.  

There are cases where it may be nearly impossible to tell if the data is incomplete or not.  For example, if a machine was opened on election day, experienced severe problems, had no votes cast on it, and was not included in the event log, it would be undetectable unless the voting system was altered; the creation of a list of machines used at each precinct would benefit this analysis.  In the case that the files are incomplete, we assume it to be a result of uploading vote data into the database at different times.  Because there are two different databases (one for the event log and one for the ballot images), the system allows for new data to be uploaded between the creation of the two files.  If there were one database used for both logs, this would reduce the problem.  On the other hand, if there were just one database, it would be nearly impossible to detect incomplete data.  

There were a number of counties' audit logs from the South Carolina 2010 elections that showed incomplete data.  Our analysis detected six counties that did not have the same set of machines in both the event log and ballot images file.  Florence County had the most inconsistencies with 65 machines that had votes cast on them according to the event log, but no ballot images.  We also saw cases where there were ballot images for votes cast on machines that did not record any events on the event log.  We also found a couple of very odd situations, such as in Sumter County, where there were two machines that were detected; one of these machines was in the event log, but not in the ballot images file, and the other machine was in the ballot images, but not in the event log.  In addition to an unusually large amount of missing data, the analysis of Florence county showed machines in both files that did not have the same number of votes cast as ballot images.  If election officials find this error when running an analysis,  they should re-upload the data to ensure a set of complete files.
