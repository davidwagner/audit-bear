\section{Conclusion}
In this study we developed a tool to analyze audit data from DRE voting machines. Our web based application, accessible to anyone, performs a variety of analyses on the audit data to detect procedural errors and system deficiencies. We replicated a previous study conducted using iVotronic audit logs collected in South Carolina after the 2010 Primary and General elections~\cite{Buell2011}. The aforementioned paper identified media devices containing votes that were not included in the certified totals. In addition, our tool can identify terminals that were not closed and their votes not uploaded to the cumulative count. This information can be very useful during the canvass process as election officials can locate the missing terminals, close them and add their votes to the election totals. 

Having performed analyses with the iVotronic logs from South Carolina, we also report statistics on polling location procedures. These statistics include: polling locations that closed late or may have experienced long lines of voters, precincts which did not report the zero tape and polling locations which used the wrong device to activate ballots on election day. Our tool can also report statistics concerning possible DRE hardware problems such as calibration issues, low battery and incorrect date and time settings. With this information election officials can improve their poll worker training or schedule voting machine repairs as needed.

%This point is going to need more work.  I'm thinking if we expand on it we might want something about Wagner's work in either our introduction or related works section.
Dr. Wagner's commissioned report ,  "Voting Systems Audit Log Study" extensively documents and evaluates many different types of audit logs produced by six different voting systems.  In the findings there were no machines that provided tools, support, or generated summary reports for analyzing audit logs.~\cite{Wagner2010}. While, the authors are very familiar with the strengths and weaknesses of the iVotronic's audit logs we would direct anyone interested in the future design of audit logs to this report.  Fully documenting the strengths and weaknesses of the iVotronic audit systems is outside the scope of this project. Our website is only the first step in creating a process for automated election auditing.  We hope that future third-party audit log  tools can build on some of this previous work to create a useful and robust solution for deriving meaningful audits directly from the logs.

We recommend that election administrators conduct routine reviews of the audit logs generated by the voting machines as they are ground truth for election disputes. By automating our analyses and making it as simple as uploading the iVotronic audit logs to a website, we believe our tool can standardize the post-election audits performed by the iVotronic system users. Our website can quickly provide intelligent feedback to election officials during the canvassing process and serve to influence future audit procedures. 
