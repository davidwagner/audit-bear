\subsection{Hardware Issues}
Election officials may be interested in identifying machines that have hardware problems, such as screen calibration issues, machines with a low battery, terminals that closed early, and machines that recorded unknown, but possibly severe events.  The first of these analyses detects machines with recurring calibration errors and machines that had recorded votes while possibly not calibrated.  By finding the events that correspond to a screen that is not calibrated and to the recalibration of that screen, we can find if votes were cast in between those times.  The second analysis regarding hardware issues looks for machines with an unusually large number of events titled \textquotedblleft Terminal shutdown - IPS Exit\textquotedblright .  We infer that these machines have a low battery because they experience a more-than-normal number of events related to the Internal Power Supply.  Additionally, our tool searches for machines that recorded a warning event about the terminal closing early.  In order for this event to occur, a trained technician must enter a password to access the service menu and make a particular selection to close the machine.  If a machine is closed in this manner during election day, there must be something wrong with it that is preventing votes from being cast correctly.  Lastly, there is a set of events that have questionable meanings, but could potentially represent hardware issues.  

Analyses such as these can help officials identify machines that may require maintenance or to be replaced.  In the case of a machine having ballots cast on it when it is not calibrated, it may not have captured the voter's intent.  Depending on the magnitude of the situation, this could cause a different outcome in the election.  If our analyses detect other hardware problems with a machine, it may not be recording votes accurately; these votes may not even appear in the event log or ballot images.  If the event log and the ballot images do not record ballots being cast, then it is nearly impossible for officials to realize votes are not being counted.  

Due to the available resources and the nature of these analyses, assumptions were made regarding the meaning of events and the severity of the situation.  Currently, there is no user manual or detailed description of the events that appear in the event log; because of this, we are not able to guarantee that the event \textquotedblleft Terminal shutdown - IPS Exit\textquotedblright means the machine has a low battery.  This assumption is also applicable to our calibration analysis and our detection of unknown warning events.  If a description of each event was available, we could be more definitive in our results and possibly implement analyses that report other useful hardware failures.    

The machines used in South Carolina have experienced frequent potential hardware issues.  For example, the combination of machines in Berkeley County experienced votes cast on a machine when the machine was not calibrated, machines with possible low batteries, and at least one machine that closed early.  Our analysis found that there were seven counties where at least one machine was possibly not calibrated when vote(s) were cast on that machine; these errors spanned 12 different polling locations.  We suggest an election official or technician inspect these machines for possible calibration issues.  We had similar findings when searching for terminals that recorded a "Warning - Terminal Closed Early" event.  There were machines with this warning in seven counties and 13 polling locations.  Terminals should not close before 7 P.M. in South Carolina on election day; for this reason, we recommend that these machines be evaluated for potential problems that would have caused early closure.  When our tool reports machines with possible low batteries, election officials should verify that the machine is working properly and does not need maintenance.  Florence County and Greenville county experienced a number of Internal Power Supply - related events; at least one machine in each precinct had 53 and 63 instances of this event, respectively.  This could be a possible indicator that the battery is running low; therefore, the election officials should take action to ensure all machines work properly in future elections.  
