\section{Background}

\subsection{Introduction to the iVotronic}
Approximately 422 jurisdictions in the United States used the ES\&S iVotronic electronic voting terminal in 2010.  A brief description of  its functionality and main system components follows:

\begin{itemize} 
\item Voting terminal. The voting terminal is a stand-alone touchscreen voting  unit. The ports available in the back of the terminal include: serial port, compact flash card slot and power supply port. The terminal is equipped with an internal battery which keeps the terminal operational during power failure periods. To comply with federal standards, at least one audio (ADA) terminal is placed in each precinct to assist the visually impaired voters.

\item Personalized Electronic Ballot (PEB). The PEB is a proprietary cartridge designed by ES\&S to operate the iVotronic terminal.  The PEB is placed in a slot located to the left of iVotronic\textquoteright s touchscreen. The terminal and the PEB communicate through the infrared port. The South Carolina counties deploy two types of PEBs to the precinct: a) the green band master PEB and b) the red band activator PEB. Both types of PEB have the same functionality, however, poll workers are trained to perform the following tasks with each PEB type.
    \begin{itemize}
    \item Master PEB.  Poll workers use the master PEB to open polls on election day. When the PEB is placed in the terminal, the touchscreen displays the precinct\textquoteright s name programmed in the PEB so that poll workers can verify the polling location information and date/time registered in the terminal\textquoteright s internal clock. If the information displayed is correct, the poll workers open the terminal for voting. The same master PEB should be used to open all terminals of the polling location. In the same fashion, the master PEB should be used to close all terminals of the polling location at the end of the voting day. When the terminal closes, it uploads  its totals onto the master PEB. The master PEB accumulates the precinct totals which are accumulated into the official tally.
    \item Activator PEB.  This PEB is used by  poll workers to activate ballots for voters. The number of activator PEBs that the election officials program for each precinct is proportional to the number of terminals and poll workers assigned to the precinct. The ratio varies depending on the jurisdiction criteria.
    \end{itemize}
\item Removable Compact Flash card (CF). The CF cards are programmed at election central and installed in the back of the voting terminal prior to precinct deployment. The CF cards contains graphic (bitmap) files read by the voting terminal during the voting process. The CF cards are also used as an external memory device: the audit log and ballot images are written to the CF card when the terminal is closed for voting. Once the polls close, the CF cards are removed from the back of the terminal and delivered to election headquarters on election night. 

\item External printer module. This module is connected to the serial port on the back of the voting terminal. The thermal printer produces the precinct zero tape and results tape. Poll workers are instructed to print the zero tape once all iVotronics of the precinct are opened for voting. In the same fashion, the results tape should be produced when all voting terminals are closed for voting on election night.
\end{itemize}

\subsection{Description of logs}
We used three iVotronic system logs to perform the analyses described in the next section. The event log (EL152.lst), ballot image file (EL155.lst) and the ES\&S election reporting manager system log (EL168a.lst).  The header of the log files identify the County's name, the type and date of the election, the date the report was generated and the election ID. The election ID is a parameter generated by the ES\&S election programming software to uniquely identify the specific election.

The event log (152.lst) lists all iVotronic terminals used on the election. The log records the terminal configuration at headquarters prior to precinct deployment which begins with the \textquotedblleft clear and test\textquotedblright of the terminal to delete previous election data from the terminal\textquoteright s memory. The log also records, in a chronological order, all relevant election day events including polls open and polls closing and the number of ballots cast.  The event log contains several columns which include: iVotronic's terminal serial number, PEB serial number, PEB type, date, time, event code and event description. An excerpt of  an event log is given in the appendix~\ref{app:el}.

The ballot image file (155.lst) contains all ballot images saved by the iVotronic terminals during the voting process. The ballot images are segregated by precinct and terminal where the votes were cast. The ballots are saved in a random order to protect the privacy of the voter. An asterisk (*) indicates the beginning of each ballot. An excerpt of a ballot image file is given in the appendix~\ref{app:bi}.

The system log listing file (EL168a.lst) tracks activity in the election reporting database since its creation at the election headquarters. Its chronological entries reflect the commands executed by the operator(s) during  pre-election testing, election night reporting and post-election canvassing. This log contains the totals accumulated in the various precincts during election night reporting, as well as any warnings or errors reported by the reporting software system during the tabulation process. The system log also tracks the uploading of the PEBs and CF cards to the central election reporting database. Manual adjustment of precinct totals are also documented in the system log file. An excerpt of a system log file is given in the appendix~\ref{app:sl}.
