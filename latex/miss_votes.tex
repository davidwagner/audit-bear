\subsection{Votes Possibly Not Uploaded}
\subsubsection{PEBs Not Uploaded}
This analysis generates a list of PEBs used to collect votes on election day. It warns the user of any PEB, master or non-master, used to close terminals, which did not had their data uploaded to the election reporting system.  The iVotronic files used by the analysis are: EL152 to search for terminal closing information and votes saved to each PEB, and EL68A for PEB upload details.

The South Carolina counties deploy two types of PEBs to each precinct on election day: a green band master PEB to open and close terminals and red band PEBs to activate ballots once the iVotronic terminals are opened for voting. The precinct procedures, dictate that a single PEB should be used to open and close all machines at a polling location. Failure to strictly follow this protocol led to problems identified in a recent study~\cite{Buell2011}.  Similar problems were experienced in Miami-Dade County during the 2002 Primary election~\cite{Mazella2002}. In that case, poll workers used two or more PEBs to open and close terminals at their precinct.  However, election officials only uploaded one of these PEBs, because they were expecting pollworkers to follow procedures and close all machines with the same PEB. As a result, the votes from some machines were not collected on election night.  Election officials were forced to spend several days at the warehouse collecting all PEBs used in the election, printing tapes of every PEB, and uploading the votes from the PEBs that were not transported to election headquarters on election night. This caused a significant delay in the reporting of election results. 

\begin{table*}
    \begin{center}
    \begin{tabular}{| c | c | c | c |}
    \hline                   
    County &PEBs used to collect votes &PEBs not uploaded &PEBs combines votes\\
    \hline
    Anderson &77 &1 &163\\
    \hline
    Colleton &36 &1 &122\\
    \hline
    Georgetown &36 &1 &92\\
    \hline
    Greenville &154 &3 &500\\
    \hline
    Horry &121 &2 &189\\
    \hline
    Richland &128 &5 &648\\
    \hline
    Sumter &60 &2 &368\\
    \hline
    \end{tabular}
    \end{center}
    \caption{PEBs not uploaded}
    \label{tab:pebs}
\end{table*}
This analysis is intended to address the procedural errors described above. The information it produces includes: the serial number of the terminals collected in the PEBs, the number of votes contained in the PEBs and the precinct's name and number. With the information the election officials can located the missing PEBs and add those votes to the aggregated count resulting in accuracy of certified totals and voter confidence.

Table~\ref{tab:pebs} summarizes the PEBs not uploaded during the General 2010 elections in South Carolina. The system file EL168a was used to identify which PEBs containing votes were not uploaded to the election reporting software. If the South Carolina counties had access to our tool during their canvass audits they could have quickly located the PEBs.

The following are some recommendations for system improvement that would make this type of analysis easier in the future.  It would be useful if the PEBs used to close terminal(s) can upload not only the total votes collected but also the serial number of the terminals it closed. Additionally, it should be possible to import a text file containing the list of iVotronic machines and master PEBs deployed to each polling location.  That list could produce a crosscheck table for verification of iVotronics and PEBs uploaded during election night reporting.

\subsubsection{Machines Not Closed}
One of the most important aspects of any election audit is ensuring that all votes are counted.  There are two main pieces of the election system that need to be analyzed to determine if votes were left out of the count.  We have discussed the first essential piece: the PEB.  The second piece of the election system that must be taken into account is the voting machine itself.  If a machine is not closed, then a PEB has not collected this terminal's data.  By checking the event log, we can determine which machines had not been closed; our tool will display the precinct name and number, and the machine serial number that was not closed.  

This analysis is important because for two reasons: it helps detect the cases where some votes were not counted and gives officials enough information to collect those votes; and also highlights cases of incomplete audit data.  Our tool will report the precinct name and number and the machine serial number. 

While these analyses will aid officials in finding votes that may have been lost, they cannot guarantee to find all uncounted votes.  Depending on the circumstances, some votes may remain uncounted because the event logs are not suitable to other auditing techniques.  In order to account for more missing votes, a list of machines used in each precinct would be extremely helpful.  With the files used by most of our analyses, we assume audit data is complete; by conducting a different analysis, we know that this is not true.  When the audit logs are incomplete we cannot account for every machine, thus we cannot ensure all votes are counted.  If there was a master list of machines used in each precinct, then there would no longer be a problem of keeping track of all machines.  

The tool we implemented has reported a few instances of machines not being closed.  There was a single machine that wasn't closed in each of the following counties: Greenville County, Horry County, and Sumter County.  These are only detected if they are closed at some point before uploading the audit data to the database in order to print the event log and ballot images.  There may be other cases that are undetectable by our analysis because a machine's audit data will not be in the event log or ballot images file if it has not been closed in the normal circumstances. 
