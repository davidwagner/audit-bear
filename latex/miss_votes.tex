\subsection{Votes Possibly Not Uploaded}
\subsubsection{PEBs Not Uploaded}
This analysis generates a list of PEBs used to collect votes on election day. It warns the user of any PEB, master or non-master, used to close terminal(s), which had their data not uploaded to the election reporting system.  The iVotronic files used by the analysis are: EL152 to search for terminal closing information and votes saved to each PEB and EL68A for PEB upload details.

The South Carolina counties deploy two types of PEBs to each precinct on election day: a Ògreen stripe masterÓ PEB to open and close terminals and Òred stripeÓ PEBs to activate ballots once the iVotronic terminals are opened for voting. The precinct procedures, dictate that a single PEB should be used to open and close all machines at a polling location. Failure to strictly follow this protocol led to problems identified in a recent study~\cite{Buell2011}.  Similar problems were experienced in Miami-Dade County during the 2002 Primary election~\cite{Mazella2002}. In that case, poll workers used two or more PEBs to open and close terminals at their precinct.  As a result, the votes from some machines were not collected on election night.  Election officials were forced to spend several days at the warehouse collecting all PEBs used in the election, printing tapes of every PEB, and uploading the votes from the PEBs that were not transported to election headquarters on election night. This caused a significant delay in the reporting of election results. 

[Document our findings]

This analysis is intended to address the current unavailability of uniform procedures for media upload verification, and to reduce the need for on-the-spot solutions to individual problems that can lead to inaccuracies in election results. Identifying PEBs containing election data which were not uploaded to the cumulative totals will allow election officials to identify and correct those problems during the canvassing process. The information concerning the PEBs not uploaded includes: the serial number of the terminal(s) collected in the PEB(s), the number of votes processed in the terminal(s) and the precinct's name and number. With the detailed information the election officials can gather the missing PEB(s) and recollect votes from terminals not included in the cumulative totals resulting in accuracy of certified totals and voter confidence.

The following are some recommendations for system improvement that would make this type of analysis easier in the future.  It would be useful if the PEBs used to close terminal(s) can upload not only the total votes collected but also the serial number of the terminals it closed. Additionally, it should be possible to import a text file containing the list of iVotronic machines and master PEBs deployed to each polling location.  That list could produce a crosscheck table for verification of iVotronics and PEBs uploaded during election night reporting.

\subsubsection{Machines Not Closed}
[To complete by Wednesday]

\subsection{Incomplete Audit Data}
In order to conduct an accurate audit of an election, the audit data must include everything that was recorded on all voting machines during the election.  One of our analyses checks both the event log and ballot images to see if the appropriate machines are present.  Not only is it important that there are ballot images for every machine with votes cast on it, but we need to verify that there is an equal number of vote cast events and ballot images per machine.  When a machine occurs in the event log with recorded votes cast on it during election day, but does not appear in the ballot images, then there must be data missing from the ballot images file.  The reverse situation reveals the opposite error- the event log is not complete.  

The importance of complete audit data lies in the accuracy of auditing elections.  In South Carolina, one of the few components of the election paper trail are the audit logs; this magnifies the importance of having complete information.  While we can identify missing information from the event log and the ballot images file, we assume the data is complete when conducting our other analyses.  If a county supplies an incomplete log, our tool's results will be less accurate than they would be otherwise.  This is problematic because there is a higher possibility for leaving anomalies undetected; missing votes may not be found and officials may not be able to identify why errors occurred during the election.  

While the audit data is clearly not always complete, there are other cases where it may be nearly impossible to tell if the data is incomplete or not.  For example, if a machine was opened on election day, experienced severe problems, had no votes cast on it, and was not included in the event log, it would be undetectable unless the voting system was altered; the creation of a list of machines used at each precinct would benefit this analysis in addition to detecting possibly missing votes.  In the case that the files are incomplete, we assume it to be a result of uploading vote data into the database at different times.  Because there are two different databases (one for the event log and one for the ballot images), the system allows for new data to be uploaded between the creation of the two files.  If there were one database used for both logs, this would reduce the problem.  On the other hand, if there were just one database, it would be nearly impossible to detect incomplete data.  

[Include table as example here].

There were a number of counties' audit logs from the South Carolina 2010 elections that showed incomplete data.  Our analysis detected six counties that did not have the same set of machines in both the event log and ballot images file.  Florence County had the most inconsistencies with 65 machines that had votes cast on them according to the event log, but had no ballot images.  We also saw cases where there were ballot images for votes cast on machines that did not record any events on the event log.  We found a couple of rarer situations, such as Sumter County's situation; there were two machines that were detected.  One of these machines was in the event log, but not in the ballot images file, and the other machine was in the ballot images, but not in the event log.  In addition to an unusually large amount of missing data, the analysis of Florence county showed machines in both files that did not have the same number of votes cast as ballot images.  If election officials find this error when running an analysis,  they should re-upload the data to ensure a set of complete files. 
