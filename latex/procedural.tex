\subsection{Procedural Errors}
Our tool can detect procedural errors and poll worker mistakes.  A few of these are: precincts that do not print zero tapes on the morning of election day; using a master PEB to activate ballots; opening and closing machines with different PEBs.  According to the South Carolina poll worker training video (citation), poll workers are required to print at least one zero tape per polling location on the morning of the election.  Using the event log, our tool checks each polling location for this event and reports the locations that did not record this event.  Another way our tool finds procedural errors is by crosschecking the master PEBs with the PEBs used to activate ballots.  Poll workers should be using non-master PEBs to activate ballots so that the PEBs do not get switched.  Along the same lines, we report incidents of opening and closing a machine with different PEBs.  A machine should be opened and closed with the same master PEB; if not, it may be more likely that this PEB does not get uploaded.   

When poll workers cancel ballots, they must select a reason why; this is another way to detect errors.  There are seven options for canceling a ballot: wrong ballot, voter left after the ballot was issued, voter left before the ballot was issued, voter request, printer problem, terminal problem, or an unspecified reason.  If there are any instances of canceling a ballot due to a printer problem, it could be an indicator of a procedural error because ballots are not printed.  In other cases, if there is a large number of a specific reason, such as having the wrong ballot, this could indicate the poll workers are repeatedly issuing the wrong ballot.  

It may be beneficial to election officials if they could detect at which locations poll workers are following the required procedures.  Procedural errors can cause many problems including lost votes, incorrect vote counts, disgruntled voters, and long lines.  If election officials are aware of the procedures that are not being followed, they could review their precinct checklist.  This will allow for more efficient audits as well as a better voting experience for voters.  

While our analyses detect an important set of errors, there are certainly many more procedures that can be analyzed.  In addition to printing zero tapes in the morning, poll workers are required to print results tapes at the end of the election; unfortunately, this is not detectable due to the way the event log is produced.  We have inferred from the event logs that the poll workers are extracting the compact flashes before printing the results tapes, therefore the event log shows no record of the event.  

Our findings reveal the improvements needed for poll worker training and for various procedures.  Although we found that pollworkers were following procedures concerning the printing of zero tapes, there were a number of counties with procedural errors.  When opening and closing a machine, the same master PEB should be used, but in 11 counties there were cases of opening and closing machines with different PEBs.  Our results showed a correlation between this error and certain precincts, where pollworkers made those mistakes repeatedly.  Colleton County had five instances of this procedural error, but four of those instances took place at one polling location; Walterboro No 4 had machines 5129946, 5133679, 5138439, and 5138563 opened with PEB 155914, but closed with PEB 155925.  This should raise a red flag to the election officials that they may need to emphasize this procedure in poll worker training.  When poll workers activate ballots for voters, they should do so with a non-master PEB; we saw two counties that had an unusually high number of violations of this procedure.  Horry County and Richland County had 22 and 32 instances of this violation, respectively.  When election officials see this result, they may wish to revise poll worker training.  Our tool also analyzes the reasons why votes were canceled, which could give insight to procedural errors.  There is likely to be a certain number of vote cancellations due to a number of reasons, but our tool will only report the machines that recorded an abnormally large amount of vote cancellations for a specific reason.  Colleton County had a machine that recorded 12 instances of vote cancellations due to a terminal problem; in this case, we would recommend the officials to inspect the machine for potential hardware problems.  A machine in Lexington County experienced an unusually large number of vote cancellations due to a "wrong ballot"; this could be result of many problems.  The machine may have a calibration issue, or there may be a procedural error in that the poll workers are repeatedly selecting the wrong ballot.
